\documentclass[]{article}
\usepackage{lmodern}
\usepackage{amssymb,amsmath}
\usepackage{ifxetex,ifluatex}
\usepackage{fixltx2e} % provides \textsubscript
\ifnum 0\ifxetex 1\fi\ifluatex 1\fi=0 % if pdftex
  \usepackage[T1]{fontenc}
  \usepackage[utf8]{inputenc}
\else % if luatex or xelatex
  \ifxetex
    \usepackage{mathspec}
  \else
    \usepackage{fontspec}
  \fi
  \defaultfontfeatures{Ligatures=TeX,Scale=MatchLowercase}
\fi
% use upquote if available, for straight quotes in verbatim environments
\IfFileExists{upquote.sty}{\usepackage{upquote}}{}
% use microtype if available
\IfFileExists{microtype.sty}{%
\usepackage{microtype}
\UseMicrotypeSet[protrusion]{basicmath} % disable protrusion for tt fonts
}{}
\usepackage[margin=1in]{geometry}
\usepackage{hyperref}
\hypersetup{unicode=true,
            pdftitle={Monte Carlo Simulation},
            pdfauthor={Yves Beutler},
            pdfborder={0 0 0},
            breaklinks=true}
\urlstyle{same}  % don't use monospace font for urls
\usepackage{color}
\usepackage{fancyvrb}
\newcommand{\VerbBar}{|}
\newcommand{\VERB}{\Verb[commandchars=\\\{\}]}
\DefineVerbatimEnvironment{Highlighting}{Verbatim}{commandchars=\\\{\}}
% Add ',fontsize=\small' for more characters per line
\usepackage{framed}
\definecolor{shadecolor}{RGB}{248,248,248}
\newenvironment{Shaded}{\begin{snugshade}}{\end{snugshade}}
\newcommand{\KeywordTok}[1]{\textcolor[rgb]{0.13,0.29,0.53}{\textbf{#1}}}
\newcommand{\DataTypeTok}[1]{\textcolor[rgb]{0.13,0.29,0.53}{#1}}
\newcommand{\DecValTok}[1]{\textcolor[rgb]{0.00,0.00,0.81}{#1}}
\newcommand{\BaseNTok}[1]{\textcolor[rgb]{0.00,0.00,0.81}{#1}}
\newcommand{\FloatTok}[1]{\textcolor[rgb]{0.00,0.00,0.81}{#1}}
\newcommand{\ConstantTok}[1]{\textcolor[rgb]{0.00,0.00,0.00}{#1}}
\newcommand{\CharTok}[1]{\textcolor[rgb]{0.31,0.60,0.02}{#1}}
\newcommand{\SpecialCharTok}[1]{\textcolor[rgb]{0.00,0.00,0.00}{#1}}
\newcommand{\StringTok}[1]{\textcolor[rgb]{0.31,0.60,0.02}{#1}}
\newcommand{\VerbatimStringTok}[1]{\textcolor[rgb]{0.31,0.60,0.02}{#1}}
\newcommand{\SpecialStringTok}[1]{\textcolor[rgb]{0.31,0.60,0.02}{#1}}
\newcommand{\ImportTok}[1]{#1}
\newcommand{\CommentTok}[1]{\textcolor[rgb]{0.56,0.35,0.01}{\textit{#1}}}
\newcommand{\DocumentationTok}[1]{\textcolor[rgb]{0.56,0.35,0.01}{\textbf{\textit{#1}}}}
\newcommand{\AnnotationTok}[1]{\textcolor[rgb]{0.56,0.35,0.01}{\textbf{\textit{#1}}}}
\newcommand{\CommentVarTok}[1]{\textcolor[rgb]{0.56,0.35,0.01}{\textbf{\textit{#1}}}}
\newcommand{\OtherTok}[1]{\textcolor[rgb]{0.56,0.35,0.01}{#1}}
\newcommand{\FunctionTok}[1]{\textcolor[rgb]{0.00,0.00,0.00}{#1}}
\newcommand{\VariableTok}[1]{\textcolor[rgb]{0.00,0.00,0.00}{#1}}
\newcommand{\ControlFlowTok}[1]{\textcolor[rgb]{0.13,0.29,0.53}{\textbf{#1}}}
\newcommand{\OperatorTok}[1]{\textcolor[rgb]{0.81,0.36,0.00}{\textbf{#1}}}
\newcommand{\BuiltInTok}[1]{#1}
\newcommand{\ExtensionTok}[1]{#1}
\newcommand{\PreprocessorTok}[1]{\textcolor[rgb]{0.56,0.35,0.01}{\textit{#1}}}
\newcommand{\AttributeTok}[1]{\textcolor[rgb]{0.77,0.63,0.00}{#1}}
\newcommand{\RegionMarkerTok}[1]{#1}
\newcommand{\InformationTok}[1]{\textcolor[rgb]{0.56,0.35,0.01}{\textbf{\textit{#1}}}}
\newcommand{\WarningTok}[1]{\textcolor[rgb]{0.56,0.35,0.01}{\textbf{\textit{#1}}}}
\newcommand{\AlertTok}[1]{\textcolor[rgb]{0.94,0.16,0.16}{#1}}
\newcommand{\ErrorTok}[1]{\textcolor[rgb]{0.64,0.00,0.00}{\textbf{#1}}}
\newcommand{\NormalTok}[1]{#1}
\usepackage{graphicx,grffile}
\makeatletter
\def\maxwidth{\ifdim\Gin@nat@width>\linewidth\linewidth\else\Gin@nat@width\fi}
\def\maxheight{\ifdim\Gin@nat@height>\textheight\textheight\else\Gin@nat@height\fi}
\makeatother
% Scale images if necessary, so that they will not overflow the page
% margins by default, and it is still possible to overwrite the defaults
% using explicit options in \includegraphics[width, height, ...]{}
\setkeys{Gin}{width=\maxwidth,height=\maxheight,keepaspectratio}
\IfFileExists{parskip.sty}{%
\usepackage{parskip}
}{% else
\setlength{\parindent}{0pt}
\setlength{\parskip}{6pt plus 2pt minus 1pt}
}
\setlength{\emergencystretch}{3em}  % prevent overfull lines
\providecommand{\tightlist}{%
  \setlength{\itemsep}{0pt}\setlength{\parskip}{0pt}}
\setcounter{secnumdepth}{0}
% Redefines (sub)paragraphs to behave more like sections
\ifx\paragraph\undefined\else
\let\oldparagraph\paragraph
\renewcommand{\paragraph}[1]{\oldparagraph{#1}\mbox{}}
\fi
\ifx\subparagraph\undefined\else
\let\oldsubparagraph\subparagraph
\renewcommand{\subparagraph}[1]{\oldsubparagraph{#1}\mbox{}}
\fi

%%% Use protect on footnotes to avoid problems with footnotes in titles
\let\rmarkdownfootnote\footnote%
\def\footnote{\protect\rmarkdownfootnote}

%%% Change title format to be more compact
\usepackage{titling}

% Create subtitle command for use in maketitle
\newcommand{\subtitle}[1]{
  \posttitle{
    \begin{center}\large#1\end{center}
    }
}

\setlength{\droptitle}{-2em}

  \title{Monte Carlo Simulation}
    \pretitle{\vspace{\droptitle}\centering\huge}
  \posttitle{\par}
    \author{Yves Beutler}
    \preauthor{\centering\large\emph}
  \postauthor{\par}
      \predate{\centering\large\emph}
  \postdate{\par}
    \date{02 Juni 2018}


\begin{document}
\maketitle

\subsection{Introduction}\label{introduction}

This document shows the Monte Carlo simulation for the stock values of
famous international companies.

\subsection{Prerequisites}\label{prerequisites}

To run this simulation you need to install \texttt{quantmod} with the
following command: \texttt{install.packages("quantmod")}. You also need
to install \texttt{fOptions} for the theoretical black-scholes pricing.

\subsubsection{Importing stock data}\label{importing-stock-data}

We fetch the stock data from the following international companies:

\begin{itemize}
\tightlist
\item
  ABB (electronics)
\item
  UBS (finances)
\item
  Novartis (pharmaceutical)
\end{itemize}

I use the quantmod library to fetch the data from the year 2017:

\begin{Shaded}
\begin{Highlighting}[]
\KeywordTok{getSymbols}\NormalTok{(companies, }\DataTypeTok{src=}\StringTok{"yahoo"}\NormalTok{, }\DataTypeTok{from=}\StringTok{"2017-01-01"}\NormalTok{, }\DataTypeTok{to=}\StringTok{"2017-12-31"}\NormalTok{)}
\end{Highlighting}
\end{Shaded}

\begin{verbatim}
## [1] "ABB" "UBS" "NVS"
\end{verbatim}

\subsubsection{ABB}\label{abb}

These are the ABB stock values of the last year (2017):

\begin{Shaded}
\begin{Highlighting}[]
\KeywordTok{chartSeries}\NormalTok{(ABB)}
\end{Highlighting}
\end{Shaded}

\includegraphics{README_files/figure-latex/unnamed-chunk-4-1.pdf}

\paragraph{Geometric Brownian Motion
(GBM)}\label{geometric-brownian-motion-gbm}

The geometric Brownian Motion is a continuous-time stochastic process in
which the logarithm of the randomly varying quantity follows a Brownian
motion or Wiener process. We can see the borders (blue) and the mean
(yellow). The mean is the average of all the 1000 possible forcasts.

\includegraphics{README_files/figure-latex/unnamed-chunk-5-1.pdf}

\paragraph{Simulation Distribution}\label{simulation-distribution}

We want to check that our values match with the lognormal distribution
for the according companies.

\includegraphics{README_files/figure-latex/unnamed-chunk-6-1.pdf}

\paragraph{Black-Scholes Option
Prices}\label{black-scholes-option-prices}

I used the following values:

\begin{itemize}
\tightlist
\item
  Stock price: 21.14\$
\item
  Interest: 5\%
\item
  Time: 3 Months
\item
  Strike: 23\$ (call), 19\$ (put)
\end{itemize}

\begin{Shaded}
\begin{Highlighting}[]
\NormalTok{stock =}\StringTok{ }\KeywordTok{as.numeric}\NormalTok{(}\KeywordTok{Op}\NormalTok{(ABB)[}\DecValTok{1}\NormalTok{])}
\NormalTok{interest =}\StringTok{ }\FloatTok{0.05}
\NormalTok{strike =}\StringTok{ }\DecValTok{23}
\NormalTok{bsCall =}\StringTok{ }\KeywordTok{GBSOption}\NormalTok{(}\StringTok{"c"}\NormalTok{, stock, strike, }\DecValTok{1}\OperatorTok{/}\DecValTok{4}\NormalTok{, interest, interest, sigmaPeak)}
\KeywordTok{cat}\NormalTok{(}\StringTok{"estimated Call Price: "}\NormalTok{,bsCall}\OperatorTok{@}\NormalTok{price)}
\end{Highlighting}
\end{Shaded}

\begin{verbatim}
## estimated Call Price:  0.1347675
\end{verbatim}

\begin{Shaded}
\begin{Highlighting}[]
\NormalTok{strike =}\StringTok{ }\DecValTok{19}
\NormalTok{bsPut =}\StringTok{ }\KeywordTok{GBSOption}\NormalTok{(}\StringTok{"p"}\NormalTok{, stock, strike, }\DecValTok{1}\OperatorTok{/}\DecValTok{4}\NormalTok{, interest, interest, sigmaPeak)}
\KeywordTok{cat}\NormalTok{(}\StringTok{"estimated Put Price: "}\NormalTok{,bsPut}\OperatorTok{@}\NormalTok{price)}
\end{Highlighting}
\end{Shaded}

\begin{verbatim}
## estimated Put Price:  0.03021871
\end{verbatim}

\subsubsection{UBS}\label{ubs}

These are the UBS stock values of the last year (2017):

\begin{Shaded}
\begin{Highlighting}[]
\KeywordTok{chartSeries}\NormalTok{(UBS)}
\end{Highlighting}
\end{Shaded}

\includegraphics{README_files/figure-latex/unnamed-chunk-8-1.pdf}

\paragraph{Geometric Brownian Motion
(GBM)}\label{geometric-brownian-motion-gbm-1}

The geometric Brownian Motion is a continuous-time stochastic process in
which the logarithm of the randomly varying quantity follows a Brownian
motion or Wiener process. We can see the borders (blue) and the mean
(yellow). The mean is the average of all the 1000 possible forcasts.

\includegraphics{README_files/figure-latex/unnamed-chunk-9-1.pdf}

\paragraph{Simulation Distribution}\label{simulation-distribution-1}

We want to check that our values match with the lognormal distribution
for the according companies.

\includegraphics{README_files/figure-latex/unnamed-chunk-10-1.pdf}

\paragraph{Black-Scholes Option
Prices}\label{black-scholes-option-prices-1}

I used the following values:

\begin{itemize}
\tightlist
\item
  Stock price: 16.22\$
\item
  Interest: 5\%
\item
  Time: 3 Months
\item
  Strike: 18\$ (call), 15\$ (put)
\end{itemize}

\begin{Shaded}
\begin{Highlighting}[]
\NormalTok{stock =}\StringTok{ }\KeywordTok{as.numeric}\NormalTok{(}\KeywordTok{Op}\NormalTok{(UBS)[}\DecValTok{1}\NormalTok{])}
\NormalTok{interest =}\StringTok{ }\FloatTok{0.05}
\NormalTok{strike =}\StringTok{ }\DecValTok{18}
\NormalTok{bsCall =}\StringTok{ }\KeywordTok{GBSOption}\NormalTok{(}\StringTok{"c"}\NormalTok{, stock, strike, }\DecValTok{1}\OperatorTok{/}\DecValTok{4}\NormalTok{, interest, interest, sigmaPeak)}
\KeywordTok{cat}\NormalTok{(}\StringTok{"estimated Call Price: "}\NormalTok{,bsCall}\OperatorTok{@}\NormalTok{price)}
\end{Highlighting}
\end{Shaded}

\begin{verbatim}
## estimated Call Price:  0.1333171
\end{verbatim}

\begin{Shaded}
\begin{Highlighting}[]
\NormalTok{strike =}\StringTok{ }\DecValTok{15}
\NormalTok{bsPut =}\StringTok{ }\KeywordTok{GBSOption}\NormalTok{(}\StringTok{"p"}\NormalTok{, stock, strike, }\DecValTok{1}\OperatorTok{/}\DecValTok{4}\NormalTok{, interest, interest, sigmaPeak)}
\KeywordTok{cat}\NormalTok{(}\StringTok{"estimated Put Price: "}\NormalTok{,bsPut}\OperatorTok{@}\NormalTok{price)}
\end{Highlighting}
\end{Shaded}

\begin{verbatim}
## estimated Put Price:  0.1240447
\end{verbatim}

\subsubsection{Novartis}\label{novartis}

These are the Novartis stock values of the last year (2017):

\begin{Shaded}
\begin{Highlighting}[]
\KeywordTok{chartSeries}\NormalTok{(NVS)}
\end{Highlighting}
\end{Shaded}

\includegraphics{README_files/figure-latex/unnamed-chunk-12-1.pdf}

\paragraph{Geometric Brownian Motion
(GBM)}\label{geometric-brownian-motion-gbm-2}

The geometric Brownian Motion is a continuous-time stochastic process in
which the logarithm of the randomly varying quantity follows a Brownian
motion or Wiener process. We can see the borders (blue) and the mean
(yellow). The mean is the average of all the 1000 possible forcasts.

\includegraphics{README_files/figure-latex/unnamed-chunk-13-1.pdf}

\paragraph{Simulation Distribution}\label{simulation-distribution-2}

We want to check that our values match with the lognormal distribution
for the according companies.

\includegraphics{README_files/figure-latex/unnamed-chunk-14-1.pdf}

\paragraph{Black-Scholes Option
Prices}\label{black-scholes-option-prices-2}

I used the following values:

\begin{itemize}
\tightlist
\item
  Stock price: 72.89\$
\item
  Interest: 5\%
\item
  Time: 3 Months
\item
  Strike: 80\$ (call), 65\$ (put)
\end{itemize}

\begin{Shaded}
\begin{Highlighting}[]
\NormalTok{stock =}\StringTok{ }\KeywordTok{as.numeric}\NormalTok{(}\KeywordTok{Op}\NormalTok{(NVS)[}\DecValTok{1}\NormalTok{])}
\NormalTok{interest =}\StringTok{ }\FloatTok{0.05}
\NormalTok{strike =}\StringTok{ }\DecValTok{80}
\NormalTok{bsCall =}\StringTok{ }\KeywordTok{GBSOption}\NormalTok{(}\StringTok{"c"}\NormalTok{, stock, strike, }\DecValTok{1}\OperatorTok{/}\DecValTok{4}\NormalTok{, interest, interest, sigmaPeak)}
\KeywordTok{cat}\NormalTok{(}\StringTok{"estimated Call Price: "}\NormalTok{,bsCall}\OperatorTok{@}\NormalTok{price)}
\end{Highlighting}
\end{Shaded}

\begin{verbatim}
## estimated Call Price:  0.3282976
\end{verbatim}

\begin{Shaded}
\begin{Highlighting}[]
\NormalTok{strike =}\StringTok{ }\DecValTok{65}
\NormalTok{bsPut =}\StringTok{ }\KeywordTok{GBSOption}\NormalTok{(}\StringTok{"p"}\NormalTok{, stock, strike, }\DecValTok{1}\OperatorTok{/}\DecValTok{4}\NormalTok{, interest, interest, sigmaPeak)}
\KeywordTok{cat}\NormalTok{(}\StringTok{"estimated Put Price: "}\NormalTok{,bsPut}\OperatorTok{@}\NormalTok{price)}
\end{Highlighting}
\end{Shaded}

\begin{verbatim}
## estimated Put Price:  0.06550017
\end{verbatim}


\end{document}
